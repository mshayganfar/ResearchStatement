\documentclass[a4paper, 10pt]{article}

\topmargin-2.0cm

\usepackage{fancyhdr}
\usepackage{pagecounting}
\usepackage[dvips]{color}

\advance\oddsidemargin-0.65in

\textheight9.2in
\textwidth6.75in
\newcommand\bb[1]{\mbox{\em #1}}
\def\baselinestretch{1.05}

\newcommand{\hsp}{\hspace*{\parindent}}
\definecolor{gray}{rgb}{0.4,0.4,0.4}

\begin{document}
\thispagestyle{fancy}

\lhead{}
\rhead{}

\renewcommand{\headrulewidth}{0pt} 
\renewcommand{\footrulewidth}{0pt} 
\fancyfoot[C]{\footnotesize \textcolor{gray}{http://users.wpi.edu/{\raise.17ex\hbox{$\scriptstyle\sim$}}mshayganfar}} 

\pagestyle{fancy}

\rhead{\textcolor{gray}{\thepage/\totalpages{}}}

\begin{small}

\begin{center}
{\LARGE \bf RESEARCH STATEMENT}\\
\vspace*{0.1cm}
{\normalsize Mahni Shayganfar (mshayganfar@wpi.edu)}
\end{center}

My current research span the areas of computational collaboration theories,
affective computing, human-robot collaboraiton, and cognitive robotics. A
common thread in my research is in developing a theory (Affective Motivational
Collaboration Theory), design of a domain-independent architecture, and the
framework which uses this architecture to provide a collabortive behavior for
robots or vritual agents. I have resorted to prominent computational
collaboration theories, i.e., SharedPlans theory, and computaitonal models of
emotions, i.e., cognitive appraisal theory to develop my own theory. Broadly
speaking, my research belongs to the area of human-robot collaboration and its
underlyign processes, a growing field which has influence in different leading
industries such as autonomous vehicles, space exploration, manufacturing, and
any situation requiring human-robot teamwork.

\subsubsection*{Background}

The construction of robots that are intelligent, collaborative problem-solving
partners is important in robotics and applications of Artificial Intelligence.
It has always been important for us to make robots better at helping us to do
whatever they are designed for. To build collaborative robots, we need to
identify the capabilities that must be added to them so that they can work with
us or other agents. As Grosz says, collaboration must be designed into systems
from the start; it cannot be patched on \cite{grosz:collaborative-systems}.
Collaboration is a special type of coordinated activity in which the
participants work together performing a task or carrying out the activities
needed to satisfy a shared goal \cite{grosz:collaboration}. 

Collaboration involves several key properties both in structural and functional
levels. For instance, most collaborative situations involve participants who
have different beliefs and capabilities; most of the time collaborators only
have partial knowledge of the process of accomplishing the collaborative
activities; collaborative plans are more than the sum of individual plans;
collaborators are required to maintain mutual beliefs about their shared goal
throughout the collaboration; they need to be able to communicate with others
effectively; they need to commit to the group activities and to their role in
it; collaborators need to commit to the success of others; they need to
reconcile between commitments to the existing collaboration and their other
activities; and they need to interpret others' actions and utterances in the
collaboration context \cite{grosz:mice-menus}. These collaboration properties
are captured by the existing computational collaboration theories.

As I mentioned, to be collaborative, partners, e.g., a robot and a human, need
to meet the specifications stipulated by collaboration theories. These theories
argue for an essential distinction between a collaboration and a simple
interaction or even a coordination in terms of commitments
\cite{grosz:shared-plans, lochbaum:collaborative-planning}. The prominent
collaboration theories are mostly based on plans and joint intentions
\cite{cohen:teamwork,grosz:plans-discourse,Litman:discourse-commonsense}, and
they were derived from the BDI paradigm developed by Bratman
\cite{bratman:intentions-plans} which is fundamentally reliant on folk
psychology \cite{ravenscroft:folk}. The two theories, Joint Intentions
\cite{cohen:teamwork} and SharedPlans \cite{grosz:plans-discourse}, have been
extensively used to examine and describe teamwork and collaboration.

\textbf{SharedPlans theory -} The SharedPlans model of collaborative
action, presented by Grosz and Sidner \cite{grosz:planning-acting,
grosz:collaboration, grosz:plans-discourse}, aims to provide the theoretical
foundations needed for building collaborative robots/agents
\cite{grosz:collaborative-systems}. SharedPlans is a general theory of
collaborative planning that requires no notion of joint intentions, accommodates
multi-level action decomposition hierarchies and allows the process of expanding
and elaborating partial plans into full plans. SharedPlans theory explains how a
group of agents can incrementally form and execute a shared plan that then
guides and coordinates their activity towards the accomplishment of a shared
goal. SharedPlans is rooted in the observation that collaborative plans are not
simply a collection of individual plans, but rather a tight interleaving of
mutual beliefs and intentions of different team members.

%Grosz and Sidner in \cite{grosz:plans-discourse} present a model of plans to
%account for how agents with partial knowledge collaborate in the construction
% of a domain plan. They are interested in the type of plans that underlie discourse
%in which the agents are collaborating in order to achieve a shared goal. They
%propose that agents are building a shared plan in which participants have a
%collection of beliefs and intentions about the actions in the plan. Agents have
%a library of how to do their actions, i.e. recipes. These recipes might be
%partially specified as to how an action is executed, or contributes to a goal.
%Then, each agent communicates their beliefs and intentions by making utterances
%about what actions they can contribute to the shared plan. This communication
%leads to the construction of a shared plan, and ultimately termination of the
%collaboration with each agent mutually believing that there exists one agent
%% who is going to execute an action in the plan, and the fact that that agent
% has intention to perform the action, and that each action in the plan
% contributes to the goal \cite{grosz:plans-discourse,lochbaum:plan-models}.

\textbf{Joint Intentions theory -} Following Bratman's guidelines, Cohen and
Levesque propose a formal approach to building artificial collaborative agents.
The Joint Intentions theory of Cohen and Levesque \cite{cohen:teamwork,
cohen:intention-commitment, cohen:persistence-intention-commitment,
cohen:intentions, levesque:acting-together} represents one of the first attempts
to establish a formal theory of collaboration, and due to its clarity and
expression, is a widely used teamwork theory. The basic idea of Joint Intentions
theory is based on individual and joint intentions (as well as commitments) to
act as a team member. Their notion of joint intention is viewed not only as a
persistent commitment of the team to a shared goal, but also implies a
commitment on part of all its members to a mutual belief about the state of the
goal. In other words, Joint Intentions theory describes how a team of agents can
jointly act together by sharing mental states about their actions while an
intention is viewed as a commitment to perform an action. A joint intention is a
shared commitment to perform an action while in a group mental state
\cite{cohen:intention-commitment}.

%In \cite{cohen:teamwork} Cohen and Levesque establish that joint intention
%cannot be defined simply as individual intention with the team regarded as an
%individual. The reason is that after the initial formation of an intention,
% team members may diverge in their beliefs and their attitudes towards the intention.
%Instead, Cohen and Levesque generalize their own definition of intention.
% First, they present a definition of individual persistent goal and individual
%intention. Then, they define analogues of these concepts by presenting mutual
%belief in place of individual belief. The definition of joint persistent goal
%requires team members to commit to informing other members, if it comes to
%believe that the shared goal is in its terminal status. As a result, in Cohen
%and Levesque's theory, a team with a joint intention is a group that shares a
%common objective and a certain shared mental state
%\cite{jarvis:teams-multiagent-systems}.

%In this theory, once an agent entered into a joint commitment with other
% agents, the agent should communicate its private beliefs with other team members if the
%agent believes that the joint goal is in its terminal status, i.e., either the
%joint goal is achieved, or it is unachievable, or irrelevant
%\cite{wilsker:study-theories}. Thus, as we mentioned above, team members are
%committed to inform other team members when they reach the conclusion that a
%goal is achievable, impossible, or irrelevant. For instance, if a robot and an
%astronaut are collaborating to install a solar panel, and the robot reaches the
%conclusion that the welding tool has deficiency, it is essential for the robot
%to have an intention to communicate with the astronaut and make this knowledge
%common. Therefore, according to this theory, in a collaboration, agents can
%count on the commitment of other members, first to the goal and then to the
%mutual belief of the status of the goal.

\textbf{STEAM -} Tambe in \cite{tambe:flexible-teamwork} argues that teamwork in
complex, dynamic, multi-agent domains requires the agents to obtain flexibility
and reusability by using integrated capabilities. Tambe created STEAM (simply, a
\textbf{S}hell \textbf{TEAM}work) based on this idea. STEAM's operationalization
in complex, real-world domains is the key in its development to addressing 
important teamwork issues. STEAM is founded on the Joint Intentions theory and
it uses joint intentions as the basic building block of teamwork while it is
informed by key concepts from SharedPlans theory.

%In summary, STEAM builds on both Joint Intention theory and SharedPlans theory
%and tries to overcome their shortcomings. Based on joint intentions, STEAM
%builds up hierarchical structures that parallel the SharedPlans theory. Hence,
%STEAM formalizes commitments by building and maintaining Joint Intentions, and
%uses SharedPlans to formulate the team's attitudes in complex tasks.

I believe the SharedPlans and Joint Intentions collaboration theories are the
most well-defined and well-established theories in computer science. I found
SharedPlans theory more convincing than the other major and subordinate
approaches, with respect to its inclusive explanation of the collaboration
structure and its association to discourse analysis which directly improves the
communicative aspects of a collaboration theory. I also understand the value of
Joint Intentions theory due to its clarity and closeness to the foundations of
collaboration concepts. These specifications of the Joint Intentions theory can
make it applicable in multi-agent system designs and human-robot collaboration.
I also consider hybrid approaches valuable, such as STEAM, if they clearly
understand drawbacks with existing theories and successfully achieve better
collaborative agents by infusing different concepts from different theories.

In my Ph.D thesis, I attempt to lay a computational framework for the theory I
have developed based on SharedPlans and Cognitive Appraisal theories. A great
deal of my work has benefited from the integration of these well-established
theories and their underlying structure.\\

2. Limitations (funnel into my thesis)

Although all these theories are well-defined and properly introduce
collaboration concepts, they mostly explain the structure of a collaboration and
they lack the underlying domain-independent processes with which collaborative
procedures could be defined more systematically and effectively in different
applications.\\

3. Introduce topic and research question \\

4. Synopsis of first part of my work \\

5. Synopsis of second part of my work \\

6. Overview of human study at the end \\

7. Verification method \\

8. Say how current research can apply to their research \\

9. Exaplain why the research is valuable \\

% Say that research work has been both theoritical and practical.

\subsubsection*{\small An Analytical Framework to Analyze Router Architectures}

In the past decade, router design has enjoyed both widespread academic interest and
commercial success. I ask the following question --- {\em Is there a common
technique, which allows us to analyze router architectures that give deterministic 
guarantees?}
I observed the existence of such a technique called constraint sets, in the course of solving 
two open problems about scaling router capacity ---

\begin{enumerate}

\item {\em Is it possible to emulate a fast ideal router, using only slower speed routers?}

\item {\em Is it possible to emulate an ideal centralized shared memory router using only distributed memories?}

\end{enumerate}

Constraint sets are a generalization of the Pigeon-hole principle applied to
routers. I showed that router design can be considered as a game where arriving
pigeons (packets) are load balanced amongst pigeon-holes (memories). 
It is surprising that the above problems can be solved \cite{pps, ppsmcast, dsm} in a simple
manner using the Pigeon-hole principle because they refute many commonly 
accepted myths about router design. 
Also the method of analysis is eye-opening because it captures the structural 
requirements of any router. 
I came up with a generic model for a class of routers called Single Buffered Routers. 
I showed how the Pigeon-hole principle can be applied in the analysis of Single Buffered
Routers that give deterministic guarantees.  Later, I extended the analysis to 
routers with two stages of buffering \cite{csets}. 
Thus our model and analysis technique
presently incorporates almost all the router architectures in use in the
core of the Internet today and shows how router capacity can be scaled in
an efficient manner.

\subsubsection*{\small Deterministic Architectures for Packet Processing}

All network equipment perform packet processing. However 
it is still not well understood, primarily due to the variety of
different processing tasks.

These tasks place a heavy demand on instruction and memory bandwidth, which 
prevents them from being implemented on general-purpose network processors.
While specific solutions exist, in most cases it is not known whether they
are optimal, whether they are complete i.e. support all necessary packet processing features and whether they
give any performance guarantees. I look at three different aspects of this problem ---

\begin{enumerate}

\item {\em Optimal and flexible packet buffers, which eliminate cache misses}

Packet buffers are built using cache hierarchies. As is well known, caching
can only give statistical guarantees on packet access time, resulting in unpredictable packet
latency.
In contrast, I proposed deterministic algorithms, which exploit the characteristics 
of memory requirements for networking to design a memory hierarchy, which eliminates cache misses. 
I showed how the optimal buffer caching algorithm can be modeled using difference equations and
used adversarial traffic patterns to show that it is optimal \cite{buffer}.
This resulting memory architecture supports the high access speeds of the cache while
having the large storage capacity of main memory, obviating the need for any special purpose memory
for networking.
Later, I showed how the cache hierarchy could be designed to allow 
flexibility in choosing any buffer access latency.
A number of router companies such as Cisco, Juniper as well as main memory manufacturers 
like Infineon, Rambus and Micron have shown interest in this technique.

\item {\em Optimal and deterministic architectures for statistics and state maintenance}

I (along with a colleague) showed using potential functions how there is 
a direct relation between the optimal architectures for buffering
and maintaining statistics counters \cite{stats}. 
Similarly I showed analytically how an algorithm, which gives deterministic 
delay bounds could be designed for maintaining connection state. 

\item {\em A complete and flexible architecture for packet classification}

Packet classification requirements vary widely.
For example, firewalls need classification on packet headers, while an
intrusion detection device requires classification of the packet content. 
Previous research has focused on being able to classify at very
high rates. In contrast, I (along with a number of colleagues) 
focused on developing a classifier, which is flexible and complete i.e. it could be programmed to
perform a number of classification tasks and give deterministic performance guarantees.
As a first step, we identified the elementary building blocks for packet
classification in terms of an abstract language. We then designed a parallel hardware architecture
to implement this 
language. This resulted in a commercial implementation of a chip set
(presently marketed by PMC-Sierra) called ClassiPI \cite{classipi}.
Among others, the ClassiPI chip set is currently in use in Cisco's
Content Services Switches.


\end{enumerate}

\subsubsection*{\small Distributed and Greedy Algorithms for Packet Switching}

Switching theory is replete with the analysis of optimal algorithms, which can give 
ideal performance, but have large complexity. What are of interest are practical algorithms that can 
be easily implemented. I answer the following open questions, which throw light on two classes of 
practical algorithms.

\begin{enumerate}
\item {\em Is there a distributed switching algorithm, which gives performance guarantees?}

The crossbar is the most common switching fabric in the core of the Internet. However,
the known switching algorithms required to give deterministic performance 
guarantees are centralized and hence have a high communication overhead.
I (along with a colleague) analyzed the feasibility of distributed algorithms for a
modification of the crossbar fabric called the buffered crossbar.
We derive analytically using combinatorial arguments and counting techniques the
conditions under which a suite of distributed algorithms can give 
both statistical and deterministic
guarantees respectively. 
Since our algorithms need only local state, do not require communication with each other, 
and can operate independently on each input and output port, they are readily implementable.
Our results show that Internet routers built using crossbars, such as Cisco routers, can 
be upgraded in a practical manner using our distributed 
algorithms on buffered crossbars and give ideal performance \cite{buffxbar}.

\item {\em When can greedy algorithms give optimal switching performance?}

Contrary to intuition, it is known in queueing theory that a greedy
switching algorithm such as the maximum size matching which
maximizes the instantaneous throughput of the switch may not
maximize the long-term switch throughput. Hence, greedy algorithms
are not in use in practice.
However greedy algorithms are of practical interest due to their low implementation complexity.
I show using Lyapunov functions the conditions under which
such algorithms give 100\% throughput \cite{msm}.

\end{enumerate}

\subsection*{Network Architecture ---  A Research Agenda}

   In the course of my research, I have noticed that the overhead (in terms
of size, power and cost) of designing networking components, which give 
performance guarantees is small. 
This is mainly due to two reasons. First, the inherent nature of 
networking makes many of these problems tractable. Second, a number of
hardware advances in Architecture, insights in Algorithms {\it \&} Combinatorics, 
as well as analysis techniques from Probability, {\it \&} Queueing Theory 
aid in the design of elegant and simple solutions.
I envisage the field of {\it Network Architecture} created from the 
ground up, building upon the foundations of a number of fields
including those mentioned above.

In the near future, I am interested in the 
principles involved in the design of basic networking 
components. These include
hardware components (e.g. scalable memories, 
network processor and co-processor architectures) and 
software techniques (e.g. network algorithms, packet processing 
techniques). 
  Simultaneously, I intend to understand how large components, which use
the above building blocks can be architected.
My research will focus on how these basic and large
components can be built in a scalable manner while maintaining 
performance guarantees. 
In particular, examples of large components that I have a keen
interest in are switches 
(e.g. packet and circuit switches, multi-service routers etc.), 
security devices (e.g. firewalls and intrusion detection 
systems), network maintenance devices (e.g. measurement,
management infrastructure) and application aware devices
(e.g. web server load balancers, proxies) etc. 

 In the future, though performance and scalability will remain key,
I also intend to look at issues such as {\it fault tolerance, graceful degradation,
reliability and uptime} of networking systems, which will become more relevant. 
I also believe that as systems become increasingly large and 
inter-dependent, {\it simplicity in design and component 
re-use} will be major factors.
Parallelism can play a key role here.

Indeed, many of our proposed solutions, involve component re-use and 
parallelism, which can aid and abet the above.

 My research will involve a good mix of futuristic and present 
day research. 
One part of my work will focus on fundamentally different proposals and
radical solutions. As an example
--- can we finally achieve real-time streaming over the Internet,
assuming that the various network components give performance guarantees?
In contrast, I intend to devote the other part of my work 
on practical systems, which have immediate relevance and impact in Industry.
I intend to work closely with a number of
researchers in related fields. Similarly, I intend to
collaborate with Industry in understanding and developing solutions
for practical problems. I believe my past experience of research
work done jointly with a number of colleagues as well as my prior record
of participation with Industry will help me achieve this. 
I am excited at the
prospect of learning, contributing, giving shape and making an
impact in this upcoming and challenging field.  


\bibliographystyle{plain}
\bibliography{mshayganfar}

\vspace{0.5cm}

\end{small}

\end{document}

