\documentclass[a4paper, 10pt]{article}

\topmargin-2.0cm

\usepackage{fancyhdr}
\usepackage{pagecounting}
\usepackage[dvips]{color}

\advance\oddsidemargin-0.65in

\textheight9.2in
\textwidth6.75in
\newcommand\bb[1]{\mbox{\em #1}}
\def\baselinestretch{1.05}

\newcommand{\hsp}{\hspace*{\parindent}}
\definecolor{gray}{rgb}{0.4,0.4,0.4}

\begin{document}
\thispagestyle{fancy}

\lhead{}
\rhead{}

\renewcommand{\headrulewidth}{0pt} 
\renewcommand{\footrulewidth}{0pt} 
\fancyfoot[C]{\footnotesize \textcolor{gray}{http://users.wpi.edu/{\raise.17ex\hbox{$\scriptstyle\sim$}}mshayganfar}} 

\pagestyle{fancy}

\rhead{\textcolor{gray}{\thepage/\totalpages{}}}

\begin{small}

\begin{center}
{\LARGE \bf RESEARCH STATEMENT}\\
\vspace*{0.1cm}
{\normalsize Mahni Shayganfar (mshayganfar@wpi.edu)}
\end{center}

My current research span the areas of computational collaboration theories,
affective computing, human-robot collaboraiton, and cognitive robotics. A
common thread in my research is in developing a theory (Affective Motivational
Collaboration Theory), design of a domain-independent architecture, and the
framework which uses this architecture to provide a collabortive behavior for
robots or vritual agents. I have resorted to prominent computational
collaboration theories, i.e., SharedPlans theory, and computaitonal models of
emotions, i.e., cognitive appraisal theory to develop my own theory. Broadly
speaking, my research belongs to the area of human-robot collaboration and its
underlyign processes, a growing field which has influence on different leading
industries such as autonomous vehicles, space exploration, manufacturing, and
any industry including situation inwhich human-robot teamwork is required.

\subsubsection*{Background - Collaboration Theories}

The construction of robots that are intelligent, collaborative problem-solving
partners is important in robotics and applications of Artificial Intelligence.
It has always been important for us to make robots better at helping us to do
whatever they are designed for. To build collaborative robots, we need to
identify the capabilities that must be added to them so that they can work with
us or other agents. As Grosz says, collaboration must be designed into systems
from the start; it cannot be patched on \cite{grosz:collaborative-systems}.
Collaboration is a special type of coordinated activity in which the
participants work together performing a task or carrying out the activities
needed to satisfy a shared goal \cite{grosz:collaboration}. 

Collaboration involves several key properties both in structural and functional
levels. For instance, most collaborative situations involve participants who
have different beliefs and capabilities; most of the time collaborators only
have partial knowledge of the process of accomplishing the collaborative
activities; collaborative plans are more than the sum of individual plans;
collaborators are required to maintain mutual beliefs about their shared goal
throughout the collaboration; they need to be able to communicate with others
effectively; they need to commit to the group activities and to their role in
it; collaborators need to commit to the success of others; they need to
reconcile between commitments to the existing collaboration and their other
activities; and they need to interpret others' actions and utterances in the
collaboration context \cite{grosz:mice-menus}. These collaboration properties
are captured by the existing computational collaboration theories.

As I mentioned, to be collaborative, partners, e.g., a robot and a human, need
to meet the specifications stipulated by collaboration theories. These theories
argue for an essential distinction between a collaboration and a simple
interaction or even a coordination in terms of commitments
\cite{grosz:shared-plans, lochbaum:collaborative-planning}. The prominent
collaboration theories are mostly based on plans and joint intentions
\cite{cohen:teamwork,grosz:plans-discourse,Litman:discourse-commonsense}, and
they were derived from the BDI paradigm developed by Bratman
\cite{bratman:intentions-plans} which is fundamentally reliant on folk
psychology \cite{ravenscroft:folk}. The two theories, Joint Intentions
\cite{cohen:teamwork} and SharedPlans \cite{grosz:plans-discourse}, have been
extensively used to examine and describe teamwork and collaboration.

\textbf{SharedPlans theory -} The SharedPlans model of collaborative
action, presented by Grosz and Sidner \cite{grosz:planning-acting,
grosz:collaboration, grosz:plans-discourse}, aims to provide the theoretical
foundations needed for building collaborative robots/agents
\cite{grosz:collaborative-systems}. SharedPlans is a general theory of
collaborative planning that requires no notion of joint intentions, accommodates
multi-level action decomposition hierarchies and allows the process of expanding
and elaborating partial plans into full plans. SharedPlans theory explains how a
group of agents can incrementally form and execute a shared plan that then
guides and coordinates their activity towards the accomplishment of a shared
goal. SharedPlans is rooted in the observation that collaborative plans are not
simply a collection of individual plans, but rather a tight interleaving of
mutual beliefs and intentions of different team members.

%Grosz and Sidner in \cite{grosz:plans-discourse} present a model of plans to
%account for how agents with partial knowledge collaborate in the construction
% of a domain plan. They are interested in the type of plans that underlie discourse
%in which the agents are collaborating in order to achieve a shared goal. They
%propose that agents are building a shared plan in which participants have a
%collection of beliefs and intentions about the actions in the plan. Agents have
%a library of how to do their actions, i.e. recipes. These recipes might be
%partially specified as to how an action is executed, or contributes to a goal.
%Then, each agent communicates their beliefs and intentions by making utterances
%about what actions they can contribute to the shared plan. This communication
%leads to the construction of a shared plan, and ultimately termination of the
%collaboration with each agent mutually believing that there exists one agent
%% who is going to execute an action in the plan, and the fact that that agent
% has intention to perform the action, and that each action in the plan
% contributes to the goal \cite{grosz:plans-discourse,lochbaum:plan-models}.

\textbf{Joint Intentions theory -} Following Bratman's guidelines, Cohen and
Levesque propose a formal approach to building artificial collaborative agents.
The Joint Intentions theory of Cohen and Levesque \cite{cohen:teamwork,
cohen:intention-commitment, cohen:persistence-intention-commitment,
cohen:intentions, levesque:acting-together} represents one of the first attempts
to establish a formal theory of collaboration, and due to its clarity and
expression, is a widely used teamwork theory. The basic idea of Joint Intentions
theory is based on individual and joint intentions (as well as commitments) to
act as a team member. Their notion of joint intention is viewed not only as a
persistent commitment of the team to a shared goal, but also implies a
commitment on part of all its members to a mutual belief about the state of the
goal. In other words, Joint Intentions theory describes how a team of agents can
jointly act together by sharing mental states about their actions while an
intention is viewed as a commitment to perform an action. A joint intention is a
shared commitment to perform an action while in a group mental state
\cite{cohen:intention-commitment}.

%In \cite{cohen:teamwork} Cohen and Levesque establish that joint intention
%cannot be defined simply as individual intention with the team regarded as an
%individual. The reason is that after the initial formation of an intention,
% team members may diverge in their beliefs and their attitudes towards the intention.
%Instead, Cohen and Levesque generalize their own definition of intention.
% First, they present a definition of individual persistent goal and individual
%intention. Then, they define analogues of these concepts by presenting mutual
%belief in place of individual belief. The definition of joint persistent goal
%requires team members to commit to informing other members, if it comes to
%believe that the shared goal is in its terminal status. As a result, in Cohen
%and Levesque's theory, a team with a joint intention is a group that shares a
%common objective and a certain shared mental state
%\cite{jarvis:teams-multiagent-systems}.

%In this theory, once an agent entered into a joint commitment with other
% agents, the agent should communicate its private beliefs with other team members if the
%agent believes that the joint goal is in its terminal status, i.e., either the
%joint goal is achieved, or it is unachievable, or irrelevant
%\cite{wilsker:study-theories}. Thus, as we mentioned above, team members are
%committed to inform other team members when they reach the conclusion that a
%goal is achievable, impossible, or irrelevant. For instance, if a robot and an
%astronaut are collaborating to install a solar panel, and the robot reaches the
%conclusion that the welding tool has deficiency, it is essential for the robot
%to have an intention to communicate with the astronaut and make this knowledge
%common. Therefore, according to this theory, in a collaboration, agents can
%count on the commitment of other members, first to the goal and then to the
%mutual belief of the status of the goal.

\textbf{STEAM -} Tambe in \cite{tambe:flexible-teamwork} argues that teamwork in
complex, dynamic, multi-agent domains requires the agents to obtain flexibility
and reusability by using integrated capabilities. Tambe created STEAM (simply, a
\textbf{S}hell \textbf{TEAM}work) based on this idea. STEAM's operationalization
in complex, real-world domains is the key in its development to addressing 
important teamwork issues. STEAM is founded on the Joint Intentions theory and
it uses joint intentions as the basic building block of teamwork while it is
informed by key concepts from SharedPlans theory.

%In summary, STEAM builds on both Joint Intention theory and SharedPlans theory
%and tries to overcome their shortcomings. Based on joint intentions, STEAM
%builds up hierarchical structures that parallel the SharedPlans theory. Hence,
%STEAM formalizes commitments by building and maintaining Joint Intentions, and
%uses SharedPlans to formulate the team's attitudes in complex tasks.

I believe the SharedPlans and Joint Intentions collaboration theories are the
most well-defined and well-established theories in computer science. I found
SharedPlans theory more convincing than the other major and subordinate
approaches, with respect to its inclusive explanation of the collaboration
structure and its association to discourse analysis which directly improves the
communicative aspects of a collaboration theory. I also understand the value of
Joint Intentions theory due to its clarity and closeness to the foundations of
collaboration concepts. These specifications of the Joint Intentions theory can
make it applicable in multi-agent system designs and human-robot collaboration.
I also consider hybrid approaches valuable, such as STEAM, if they clearly
understand drawbacks with existing theories and successfully achieve better
collaborative agents by infusing different concepts from different theories.

\subsubsection*{Background - Cognitive Appraisal Theory}

According to Picard \cite{picard:affective-computing}, the term affective
computing encapsulates a new approach in Artificial Intelligence, to build
computers that show human affection. Studies show that the decision making of
humans is not always logical \cite{GrossbergGutowski:affect-cognition}, and in
fact, not only is pure logic not enough to model human intelligence, but it also
shows failures when applied in artificial intelligence systems
\cite{dreyfus:artificial-critique}. Emotions impact fundamental parts of
cognition including perception, memory, attention and reasoning
\cite{clore:judgement-regulation}. This impact is caused by the information
emotions carry about the environment and event values. If we want robots and
virtual agents to be more believable and efficient partners for humans, we must
consider the personal and social functionalities and characteristics of
emotions; this will enable our robots to coexist with humans, who are emotional
beings.

\textbf{Cognitive Appraisal Theory -} There are different types of computational
theories of emotion. These theories differ in the type of relationships between
their components and whether a particular component plays a crucial role in an
individual emotion. Appraisal theories of emotion were first formulated by Arnold
\cite{arnold:emotion-personality} and Lazarus \cite{lazarus:emotion-adaptation}
and then were actively developed in the early 80s by Ellsworth and Scherer and
their students \cite{roseman:appraisal-theory,
sander:systems-approach-appraisal, scherer:nature-function-emotion,
scherer:emotions-emergent, scherer:appraisal-processes}. The emotional
experience is the experience of a particular situation \cite{frijda:emotions}.
Appraisal theory describes the cognitive process by which an individual
evaluates the situation in the environment with respect to the individual's
well-being and triggers emotions to control internal changes and external
actions. According to this theory, appraisals are separable antecedents of
emotion, that is, the individual first evaluates the environment and then feels
an appropriate emotion \cite{scherer:appraisal-processes}. The appraisal
procedure begins with the evaluation of the environment according to the
internalized goals and is based on systematic assessment of several elements
\cite{scherer:sequential-appraisal-process}. The outcome of this process
triggers the appropriate emotions.\\


In my Ph.D thesis, I attempt to lay a computational framework for the theory I
have developed based on SharedPlans and Cognitive Appraisal theories. A great
deal of my work has benefited from the integration of these well-established
theories and their underlying structure.

\subsubsection*{Limitations}

Although all the existing collaboration theories are well-defined and properly
introduce collaboration concepts, they mostly explain the structure of a
collaboration and they lack the underlying domain-independent processes with
which collaborative procedures could be defined more systematically and
effectively in different applications.\\

3. Introduce topic and research question \\

I believe that the evaluative role of emotions as a part of cognitive processes
helps an agent to perform appropriate behaviors during a collaboration. It is
important to think about the underlying cognitive processes of the collaborators
in order to have a better understanding of the role of emotions. To work jointly
in a coordinated activity, participants (collaborators) act based on their own
understanding of the world and the anticipated mental states of the counterpart;
this understanding is reflected in their collaborative behaviors. Emotions are
pivotal in the collaboration context, since their regulatory and motivative
roles enhance an individual's autonomy and adaptation as well as his/her
coordination and communication competencies in a dynamic, uncertain and
resource-limited environment. The collaborative behavior of the individuals can
also be influenced by the tasks contributing towards a shared goal. Some tasks
may be inherently insignificant, boring, unpleasant or arduous for a
collaborator. Thus, knowing how to externally motivate other collaborator to
perform such tasks becomes an essential skill for a participant in a successful
collaboration. Such knowledge enables an individual to lead his collaborator to
internalize the responsibility and sense of value for an externally motivated
task.

\subsubsection*{Current Research}

In my Ph.D. thesis, I have developed the Affective Motivational Collaboration
Theory and the associated computational model that will enhance the performance
and effectiveness of collaboration between robots and humans. This theory
explains the functions of emotions in a dyadic collaboration and shows how
affective mechanisms can coordinate social interactions by anticipating other's
emotions, beliefs and intentions. This theory also specifies the influence of
the underlying collaboration processes on appraisals. Affective Motivational
Collaboration Theory elucidates the role of motives as goal-driven
emotion-regulated constructs with which an agent can form new beliefs and
intentions to cope with internal and external events. An important contribution
of this work is to elucidate how motives are involved not only in the appraisal
and coping processes, but how they also serve as a bridge between appraisal
processes and the collaboration structure. I will validate my theory using my
computational framework in the context of a human-robot collaboration.

5. Synopsis of second part of my work \\

6. Overview of human study at the end \\

7. Verification method \\

8. Say how current research can apply to their research \\

9. Exaplain why the research is valuable

\subsubsection*{A Research Agenda}

In the course of my research, I have noticed that \ldots

In the near future, I am interested in the 
principles involved in the design of basic networking 
components. These include
hardware components (e.g. scalable memories, 
network processor and co-processor architectures) and 
software techniques (e.g. network algorithms, packet processing 
techniques). 
  Simultaneously, I intend to understand how large components, which use
the above building blocks can be architected.
My research will focus on how these basic and large
components can be built in a scalable manner while maintaining 
performance guarantees. 
In particular, examples of large components that I have a keen
interest in are switches 
(e.g. packet and circuit switches, multi-service routers etc.), 
security devices (e.g. firewalls and intrusion detection 
systems), network maintenance devices (e.g. measurement,
management infrastructure) and application aware devices
(e.g. web server load balancers, proxies) etc. 

 In the future, though performance and scalability will remain key,
I also intend to look at issues such as {\it fault tolerance, graceful degradation,
reliability and uptime} of networking systems, which will become more relevant. 
I also believe that as systems become increasingly large and 
inter-dependent, {\it simplicity in design and component 
re-use} will be major factors.
Parallelism can play a key role here.

Indeed, many of our proposed solutions, involve component re-use and 
parallelism, which can aid and abet the above.

 My research will involve a good mix of futuristic and present 
day research. 
One part of my work will focus on fundamentally different proposals and
radical solutions. As an example
--- can we finally achieve real-time streaming over the Internet,
assuming that the various network components give performance guarantees?
In contrast, I intend to devote the other part of my work 
on practical systems, which have immediate relevance and impact in Industry.
I intend to work closely with a number of
researchers in related fields. Similarly, I intend to
collaborate with Industry in understanding and developing solutions
for practical problems. I believe my past experience of research
work done jointly with a number of colleagues as well as my prior record
of participation with Industry will help me achieve this. 
I am excited at the
prospect of learning, contributing, giving shape and making an
impact in this upcoming and challenging field.  


\bibliographystyle{plain}
\bibliography{mshayganfar}

\vspace{0.5cm}

\end{small}

\end{document}

