\documentclass[a4paper, 11pt]{article}

\topmargin-2.0cm

\usepackage{fancyhdr}
\usepackage{pagecounting}
\usepackage[dvips]{color}

\advance\oddsidemargin-0.65in

\textheight9.2in
\textwidth6.25in
\newcommand\bb[1]{\mbox{\em #1}}
\def\baselinestretch{1.05}

\newcommand{\hsp}{\hspace*{\parindent}}
\definecolor{gray}{rgb}{0.4,0.4,0.4}

\begin{document}
\thispagestyle{fancy}

\lhead{}
\rhead{}

\renewcommand{\headrulewidth}{0pt} 
\renewcommand{\footrulewidth}{0pt} 
\fancyfoot[C]{\footnotesize \textcolor{gray}{http://users.wpi.edu/{\raise.17ex\hbox{$\scriptstyle\sim$}}mshayganfar}} 

\pagestyle{fancy}

\rhead{\textcolor{gray}{\thepage/\totalpages{}}}

\begin{small}

\begin{center}
{\LARGE \bf RESEARCH STATEMENT}\\
\vspace*{0.1cm}
{\normalsize Mahni Shayganfar (mshayganfar@wpi.edu)}
\vspace{6mm}\\
{\Large \bf Affect-Driven Cognitive Human Performance \\Evaluation \&
Adaptation}\\
\end{center}

My current research spans the areas of computational collaboration theories,
affective computing, human-robot collaboration, and cognitive robotics. Common
threads in my research are the development of Affective Motivational
Collaboration Theory, design of a domain-independent architecture, as well as
the design of the framework which uses this architecture. The goal is to provide
a collaborative behavior for robots or virtual agents based on this
architecture. I have built on existing computational collaboration theories,
i.e., SharedPlans theory, and computational models of emotions, i.e., cognitive
appraisal theory to develop my own theory. Broadly speaking, my research belongs
to the area of human-robot collaboration and its underlying processes, a growing
field which has influence on different leading industries such as autonomous
vehicles, space exploration, manufacturing, and any industry situation in which
human-robot teamwork is required.

Improving the human-robot teamwork performance not only depends on the robot's
computational and physical capabilities, but it also depends on the human's
performance and the ability of the robot to maintain this performance according
to human's cognitive state. It is crucial to understand that the human's
performance dynamically changes with respect to the internal and external
events. The internal events comprise changes in mental state, e.g., violation of
a strong belief, occurance of ambivalent intentions, and the blockage of an
urgent motive. The external events can also influence one's performance, e.g.,
abrupt changes of a task's timing, failure of high priority tasks, and delay in
task duration. A collaborative robot requires to evaluate the existing data with
respect to the human's cognitive model, adopt corresponding and appropriate
behaviors, and adapt its own actions to maintain human's performance in task
execution.

\vspace*{-4mm}
\section*{Background and Motivation}
\vspace*{-2mm}
The construction of robots that are intelligent, collaborative problem-solving
partners is important in robotics and applications of Artificial Intelligence.
It has always been important for us to make robots better at helping us to do
whatever they are designed for. To build collaborative robots, we need to
identify the capabilities that must be added to them so that they can work with
us or other agents. Collaboration must be designed into systems from the start;
it cannot be patched on \cite{grosz:collaborative-systems}. Collaboration is a
type of coordinated activity in which the participants work together performing
a task or carrying out the activities needed to satisfy a shared goal
\cite{grosz:collaboration}.
\vspace*{-4mm}
\subsubsection*{Collaboration Theories}
Collaboration involves several key properties both a structural and functional
levels. For instance, most collaborative situations involve participants who
have different beliefs and capabilities; most of the time collaborators only
have partial knowledge of the process of accomplishing the collaborative
activities; collaborative plans are more than the sum of individual plans;
collaborators are required to maintain mutual beliefs about their shared goal
throughout the collaboration; they need to be able to communicate with others
effectively; they need to commit to the group activities and to their role in
it; collaborators need to commit to the success of others; they need to
reconcile between commitments to the existing collaboration and their other
activities; and they need to interpret others' actions and utterances in the
collaboration context \cite{grosz:mice-menus}. These collaboration properties
are captured by the existing computational collaboration theories.

As I mentioned, to be collaborative, partners, e.g., a robot and a human, need
to meet the specifications stipulated by collaboration theories. The prominent
collaboration theories are mostly based on plans and joint intentions
\cite{cohen:teamwork,grosz:plans-discourse,Litman:discourse-commonsense}, and
they were derived from the BDI paradigm developed by Bratman
\cite{bratman:intentions-plans} which is fundamentally reliant on folk
psychology \cite{ravenscroft:folk}. The two theories, Joint Intentions
\cite{cohen:teamwork} and SharedPlans \cite{grosz:plans-discourse}, have been
extensively used to examine and describe teamwork and collaboration. The
SharedPlans model of collaborative action \cite{grosz:planning-acting,
grosz:collaboration, grosz:plans-discourse}, aims to provide the theoretical
foundations needed for building collaborative robots/agents
\cite{grosz:collaborative-systems}. SharedPlans is a general theory of
collaborative planning that requires no notion of joint intentions, accommodates
multi-level action decomposition hierarchies and allows the process of expanding
and elaborating partial plans into full plans. SharedPlans theory explains how a
group of agents can incrementally form and execute a shared plan that then
guides and coordinates their activity towards the accomplishment of a shared
goal.

%Grosz and Sidner in \cite{grosz:plans-discourse} present a model of plans to
%account for how agents with partial knowledge collaborate in the construction
% of a domain plan. They are interested in the type of plans that underlie discourse
%in which the agents are collaborating in order to achieve a shared goal. They
%propose that agents are building a shared plan in which participants have a
%collection of beliefs and intentions about the actions in the plan. Agents have
%a library of how to do their actions, i.e. recipes. These recipes might be
%partially specified as to how an action is executed, or contributes to a goal.
%Then, each agent communicates their beliefs and intentions by making utterances
%about what actions they can contribute to the shared plan. This communication
%leads to the construction of a shared plan, and ultimately termination of the
%collaboration with each agent mutually believing that there exists one agent
%% who is going to execute an action in the plan, and the fact that that agent
% has intention to perform the action, and that each action in the plan
% contributes to the goal \cite{grosz:plans-discourse,lochbaum:plan-models}.

The Joint Intentions theory of Cohen and Levesque \cite{cohen:teamwork,
cohen:intention-commitment, cohen:persistence-intention-commitment,
cohen:intentions, levesque:acting-together} represents one of the first attempts
to establish a formal theory of collaboration, and due to its clarity and
expression, is a widely used teamwork theory. The basic idea of Joint Intentions
theory is based on individual and joint intentions (as well as commitments) to
act as a team member. Their notion of joint intention is viewed not only as a
persistent commitment of the team to a shared goal, but also implies a
commitment on part of all its members to a mutual belief about the state of the
goal. In other words, Joint Intentions theory describes how a team of agents can
jointly act together by sharing mental states about their actions while an
intention is viewed as a commitment to perform an action 
\cite{cohen:intention-commitment}.

%In \cite{cohen:teamwork} Cohen and Levesque establish that joint intention
%cannot be defined simply as individual intention with the team regarded as an
%individual. The reason is that after the initial formation of an intention,
% team members may diverge in their beliefs and their attitudes towards the intention.
%Instead, Cohen and Levesque generalize their own definition of intention.
% First, they present a definition of individual persistent goal and individual
%intention. Then, they define analogues of these concepts by presenting mutual
%belief in place of individual belief. The definition of joint persistent goal
%requires team members to commit to informing other members, if it comes to
%believe that the shared goal is in its terminal status. As a result, in Cohen
%and Levesque's theory, a team with a joint intention is a group that shares a
%common objective and a certain shared mental state
%\cite{jarvis:teams-multiagent-systems}.

%In this theory, once an agent entered into a joint commitment with other
% agents, the agent should communicate its private beliefs with other team members if the
%agent believes that the joint goal is in its terminal status, i.e., either the
%joint goal is achieved, or it is unachievable, or irrelevant
%\cite{wilsker:study-theories}. Thus, as we mentioned above, team members are
%committed to inform other team members when they reach the conclusion that a
%goal is achievable, impossible, or irrelevant. For instance, if a robot and an
%astronaut are collaborating to install a solar panel, and the robot reaches the
%conclusion that the welding tool has deficiency, it is essential for the robot
%to have an intention to communicate with the astronaut and make this knowledge
%common. Therefore, according to this theory, in a collaboration, agents can
%count on the commitment of other members, first to the goal and then to the
%mutual belief of the status of the goal.

There are also hybrid collaboration theories which borrow collaboration
concepts from both SharedPlans and Joint Intentions theories. For instance,
STEAM \cite{tambe:flexible-teamwork} (simply, a Shell TEAMwork) is a hybrid
collaboration theory which its operationalization in complex, real-world domains
is the key in its development to addressing important teamwork issues. STEAM is
founded on the Joint Intentions theory and it uses joint intentions as the basic
building block of teamwork while it is informed by key concepts from SharedPlans
theory.

%In summary, STEAM builds on both Joint Intention theory and SharedPlans theory
%and tries to overcome their shortcomings. Based on joint intentions, STEAM
%builds up hierarchical structures that parallel the SharedPlans theory. Hence,
%STEAM formalizes commitments by building and maintaining Joint Intentions, and
%uses SharedPlans to formulate the team's attitudes in complex tasks.

I believe the SharedPlans and Joint Intentions collaboration theories are the
most well-defined and well-established theories in computer science. I found
SharedPlans theory more convincing than the other major and subordinate
approaches, with respect to its inclusive explanation of the collaboration
structure and its association to discourse analysis which directly improves the
communicative aspects of a collaboration theory. I also understand the value of
Joint Intentions theory due to its clarity and closeness to the foundations of
collaboration concepts. These specifications of the Joint Intentions theory can
make it applicable in multi-agent system designs and human-robot collaboration.
I also consider hybrid approaches valuable, such as STEAM, if they clearly
understand drawbacks with existing theories and successfully achieve better
collaborative agents by infusing different concepts from different theories.
\vspace*{-8mm}
\subsubsection*{Cognitive Appraisal Theory}

The term affective computing encapsulates a new approach in Artificial
Intelligence, to build computers that show human affection
\cite{picard:affective-computing}. Studies show that the decision making of
humans is not always logical \cite{GrossbergGutowski:affect-cognition}, and in
fact, not only is pure logic not enough to model human intelligence, but it also
shows failures when applied in artificial intelligence systems
\cite{dreyfus:artificial-critique}. Emotions impact fundamental parts of
cognition including perception, memory, attention and reasoning
\cite{clore:judgement-regulation}. This impact is caused by the information
emotions carry about the environment and event values. If we want robots and
virtual agents to be more believable and efficient partners for humans, we must
consider the personal and social functionalities and characteristics of
emotions; this will enable our robots to coexist with humans, who are emotional
beings.

There are different types of computational theories of emotion. These theories
differ in the type of relationships between their components and whether a
particular component plays a crucial role in an individual emotion. Appraisal
theory describes the cognitive process by which an individual evaluates the
situation in the environment with respect to the individual's well-being and
triggers emotions to control internal changes and external actions. According to
this theory, appraisals are separable antecedents of emotion, that is, the
individual first evaluates the environment and then feels an appropriate emotion
\cite{scherer:appraisal-processes}. The appraisal procedure begins with the
evaluation of the environment according to the internalized goals and is based
on systematic assessment of several elements
\cite{scherer:sequential-appraisal-process}. The outcome of this process
triggers the appropriate emotions.\\

In my Ph.D thesis, I lay out a computational framework for the theory I have
developed based on SharedPlans and Cognitive Appraisal theories. A great deal of
my work has benefited from the integration of these well-established theories
and their underlying structure.

\subsubsection*{Limitations of Existing Theories}

There are several well-developed
cognitive architectures, e.g., Soar \cite{laird:soar} and ACT-R
\cite{anderson:act-r}, each with different approaches to defining the basic
cognitive and perceptual operations. There have also been efforts to integrate
affect into these architectures \cite{marinier:behavior-emotion}. In general,
however, these cognitive architectures do not focus on processes to specifically
produce affect-regulated goal-driven collaborative behaviors. At the same time,
existing collaboration theories, e.g. SharedPlans theory
\cite{grosz:plans-discourse}, Joint Intentions \cite{cohen:teamwork}, and STEAM
\cite{tambe:flexible-teamwork}, focus on describing the structure of a
collaboration in terms of fundamental mental states, e.g., mutual beliefs or
intentions. Although all the existing collaboration theories are well-defined
and properly introduce collaboration concepts, they mostly explain the structure
of a collaboration and they lack the underlying domain-independent processes
with which collaborative procedures could be defined more systematically and
effectively in different applications. Therefore, it is crucial to investigate
the cognitive processes involved in a collaboration in the context of a
cognitive architecture to be able to describe the underlying processes, their
relationships, and their influences on each other.\\

I believe that the evaluative role of emotions as a part of cognitive processes
helps an agent to perform appropriate behaviors during a collaboration. It is
important to think about the underlying cognitive processes of the collaborators
in order to have a better understanding of the role of emotions. To work jointly
in a coordinated activity, participants (collaborators) act based on their own
understanding of the world and the anticipated mental states of the counterpart;
this understanding is reflected in their collaborative behaviors. Emotions are
pivotal in the collaboration context, since their regulatory and motivative
roles enhance an individual's autonomy and adaptation as well as his/her
coordination and communication competencies in a dynamic, uncertain and
resource-limited environment. The collaborative behavior of the individuals can
also be influenced by the tasks contributing towards a shared goal. Some tasks
may be inherently insignificant, boring, unpleasant or arduous for a
collaborator. Thus, knowing how to externally motivate other collaborator to
perform such tasks becomes an essential skill for a participant in a successful
collaboration. Such knowledge enables an individual to lead his collaborator to
internalize the responsibility and sense of value for an externally motivated
task.

\subsubsection*{Current Research}

In my Ph.D. thesis, I have developed Affective Motivational Collaboration Theory
and the associated computational model that will enhance the performance and
effectiveness of collaboration between robots and humans. This theory explains
the functions of emotions in a dyadic collaboration and shows how affective
mechanisms can coordinate social interactions by anticipating other's emotions,
beliefs and intentions. This theory also specifies the influence of the
underlying collaboration processes on appraisals. Affective Motivational
Collaboration Theory elucidates the role of motives as goal-driven
emotion-regulated constructs with which an agent can form new beliefs and
intentions to cope with internal and external events. An important contribution
of this work is to elucidate how motives are involved not only in the appraisal
and coping processes, but how they also serve as a bridge between appraisal
processes and the collaboration structure. I am validating my theory using my
computational framework in the context of a human-robot collaboration.

I have investigated how affect-driven mechanisms in my Affective Motivational
Collaboration Theory can be involved to successfully maintain a dyadic
collaboration. My research is based on two hypothetical collaboration examples
of \textit{task delegation} and \textit{agreeing on a shared goal} for both
success and failure of each scenario because of the emotion-awareness and
emotion-ignorance behavior of the robot. I have implemented the rules associated
with these examples using JESS (Java Expert System Shell) which is a rule engine
for the Java platform. In my current implementation I have categorized the rules
in different modules associated with the mechanisms and the underlying processes
in Affective Motivational Collaboration Theory. The outcome of this part of my
work is currently under review as a journal publication. I am implementing
algorithms for each individual mechanism to be able to automatically generate
the required facts within each mechanism to fire the existing rules. Ultimately,
I am going to provide a platform which operates based on the collaboration
structure, and employs affect-driven processes such as the appraisal processes
in \cite{marsella:ema-process-model} to enable a robot to obtain and demonstrate
collaborative behaviors.

% In positive cases, two walkthrough examples are provided to show successful
% progress of a collaboration in each scenario. In these examples collaboration is
% concluded by the shared goal achievement because of the emotion-aware behavior
% of a robot. Whereas, the negative cases, illustrate how emotion-ignorant
% behavior of a robot can cause failure of a collaboration in the same examples. 

As another crucial part of my research, I have developed domain-independent
algorithms for the appraisal of a collaborative environment. These algorithms
compute the value of the four appraisal variables: \textit{relevance,
desirability, expectedness} and \textit{controllability}. These algorithms use a
generalized collaboration structure to compute their outcome in a dyadic
collaboration. The performance of individual algorithms will be verified based
on the analysis of the answers human subjects provide to our questionnaire. All
of the questions have been prepared based on the same task models provided to
our framework. I will finalize my Ph.D. thesis by evaluating the overall
architecture using a robot simulator to successfully and efficiently (in terms
of time and number of errors) collaborate with human subjects.

\subsubsection*{Future Research Directions}

Collaborative robots are becoming an integral part of humans' environment to
accomplish their industrial and household tasks. In these environments humans
are involved in robots' operations and decision making processes. This
involvement of humans influences the efficiency of robots' interaction and
performance, and makes them dependent on the humans' cognitive abilities and
mental states. Similarly, humans' performance can be maintained using robot's
adaptive behavior. The evaluative nature of affective processes can assist a
robot to perform adaptive behaviors with respect to the current goal. Goal
management, alarm mechanism and action selection are some examples of the
functions that a system with affect-driven processes offers.

Performance evaluation of a human requires perception of the human's physical
and mental states. This perception can be performed based on available
bio-information, as well as verbal (e.g., utterances) and non-verbal (e.g.,
facial expressions) behaviors. However, a robot requires cognitive mechanisms to
evaluate and adapt its own behavior to change the human's performance on the
given task and environment. My Ph.D. thesis provides fundamental affect-driven
mechanisms and defines their relationship within a cognitive architecture. These
mechanisms can be involved from initial evaluation of the input data to higher
level processes such as action selection or alarm mechanism. All these functions
can improve a robot's awareness and self-synchronization capabilities during the
interaction. For instance, if the appraisal of the input data reveals
controllability (an appraisal variable) of an event, the robot can choose an
appropriate action to make changes to the status of the interaction. Ultimately,
these changes can improve human's performance in a task execution. I would like
to apply my knowledge I attained in my research to develop new algorithms to
evaluate performance of a human and adapt the robot's behavior respectively.



\bibliographystyle{plain}
\bibliography{mshayganfar}

\vspace{0.5cm}

\end{small}

\end{document}